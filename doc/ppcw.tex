\documentclass[twoside,11pt]{article}
\PassOptionsToPackage{hyphens}{url}
\usepackage{jmlr2e}
\usepackage{amsmath}
\usepackage[toc,page]{appendix}
\usepackage[table]{xcolor}
\usepackage[marginparsep=30pt]{geometry}
\usepackage{stmaryrd}
\usepackage{algorithm}
\usepackage{algorithmic}
\usepackage{tikz}
\usepackage{pgfplots}
\usepackage{tabu}
\usepackage{longtable}
\usepackage{tabularx}
\usepackage{listings}
\usepackage{fancyref}
\usepackage{relsize}
\usepackage{float}
\usepackage{subcaption}
\usepackage{diagbox}

\usetikzlibrary{%
    arrows,
    arrows.meta,
    decorations,
    backgrounds,
    positioning,
    fit,
    petri,
    shadows,
    datavisualization.formats.functions,
    calc,
    shapes,
    shapes.multipart,
    matrix,
    plotmarks
}

\usepgfplotslibrary{fillbetween, statistics}

\pgfplotsset{
  compat=1.3,
  every non boxed x axis/.style={
  enlarge x limits=false,
  x axis line style={}%-stealth},
  },
  every boxed x axis/.style={},
  every non boxed y axis/.style={
  enlarge y limits=false,
  y axis line style={}%-stealth},
  },
  every boxed y axis/.style={},
}

\def\perc{\texttt{perco\-late}}
\def\v{\texttt{v0.1.0}}

\def\titl{Performance Programming Coursework:
  Serial Optimization of a Molecual Dynamics Program}

\title{\titl}

\author{}

\ShortHeadings{B160509}{B160509}
\firstpageno{1}


\begin{document}

\maketitle

\begin{abstract}
\end{abstract}

\begin{keywords}
Scientific programming, serial optimization, molecular dynamics,
Fortran
\end{keywords}

\section{Introduction} % {{{

This paper documents the serial performance optimization conducted for
a molecular dynamics program written in the Fortran programming
language.
The program reads data from $n$ molecules as its input and iterates
a predefined amount of steps.
Each step the data of the molecules (e.g.\ force, position in a three
dimensional space, velocity) is updated.
The program counts collisions between molecules and writes the
data for each molecule of the simulation out at certain, also
predefined, intervals.
The original version of the program is not optimized for computational
efficiency.

This paper describes, documents and discusses the process of
optimizing the original version of the program serially.
The program was optimized for the Cirrus supercomputer, a tier-2
UK national supercomputer of the engineering and physical sciences
research council, which is hosted and maintained by the EPCC
\citep{cirrus}.
The compiler used was Intel's Fortran compiler \texttt{ifort}, version
$18.0.5$ \citep{ifort}.
The conducted optimizations range from choosing the appropriate
compiler flags over rewriting performance critical sections of the
program to hardware specific optimizations, like leveraging
vectorization and cache optimizations.
Since raw performance benefits are not all that is important for
writing well performing and good programs, the maintainability and
portability of the program are also looked at and discussed.

% TODO: hint on the results

This paper continues in Section~\ref{sec:md} with describing the
molecular dynamics program in detail.
Section~\ref{sec:setup} describes Cirrus and how the correctness of
the program is tested with a regression test suite.
Also the benchmark suite used for assessing the performance benefits
of an optimization is outlined.
Section~\ref{sec:opt} lists, describes and discusses all
optimizations tested with their performance benefits.
Afterwards, the results are discussed in Section~\ref{sec:dis}.
At least a conclusion is given in Section~\ref{sec:con}.
% }}}

\section{Molecular Dynamics Program} % {{{
\label{sec:md}

This section describes what the molecular dynamics program, which is
optimized, does.
% describe the physics stuff here

% }}}

\section{Setup} % {{{
\label{sec:setup}

% Cirrus

% test suite, benchmark suite

% }}}

\section{Optimizations} % {{{
\label{sec:opt}

% }}}

\section{Discussion} % {{{
\label{sec:dis}

% }}}

\section{Conclusion} % {{{
\label{sec:con}

% }}}

\bibliography{library.bib}

\end{document}
