\documentclass[twoside,11pt]{article}
\PassOptionsToPackage{hyphens}{url}
\usepackage{jmlr2e}
\usepackage{amsmath}
\usepackage[toc,page]{appendix}
\usepackage[table]{xcolor}
\usepackage[marginparsep=30pt]{geometry}
\usepackage{stmaryrd}
\usepackage{algorithm}
\usepackage{algorithmic}
\usepackage{tikz}
\usepackage{pgfplots}
\usepackage{tabu}
\usepackage{longtable}
\usepackage{tabularx}
\usepackage{listings}
\usepackage{fancyref}
\usepackage{relsize}
\usepackage{float}
\usepackage{subcaption}
\usepackage{diagbox}

\usetikzlibrary{%
    arrows,
    arrows.meta,
    decorations,
    backgrounds,
    positioning,
    fit,
    petri,
    shadows,
    datavisualization.formats.functions,
    calc,
    shapes,
    shapes.multipart,
    matrix,
    plotmarks
}

\usepgfplotslibrary{fillbetween, statistics}

\pgfplotsset{
  compat=1.3,
  every non boxed x axis/.style={
  enlarge x limits=false,
  x axis line style={}%-stealth},
  },
  every boxed x axis/.style={},
  every non boxed y axis/.style={
  enlarge y limits=false,
  y axis line style={}%-stealth},
  },
  every boxed y axis/.style={},
}

\lstset{
    language=C,
    breaklines=true,
    keepspaces=true,
    %keywordstyle=\bfseries\color{green!70!black},
    %basicstyle=\ttfamily\color{black},
    %commentstyle=\itshape\color{purple},
    %identifierstyle=\color{blue},
    %stringstyle=\color{orange},
    showstringspaces=false,
    %rulecolor=\color{black},
    tabsize=2,
    escapeinside={\%*}{*\%},
}

\def\perc{\texttt{perco\-late}}
\def\v{\texttt{v0.1.0}}

\def\titl{Performance Programming Coursework:
  Serial Optimization of a Molecular Dynamics Program}

\title{\titl}

\author{}

\ShortHeadings{B160509}{B160509}
\firstpageno{1}


\begin{document}

\maketitle

\begin{abstract}
  This paper documents the serial optimization of a molecular dynamics
  program written in the Fortran programming language.
  The original version of the program took 1270 seconds to complete
  a simulation on the Cirrus supercomputer.
  The final version took 26 seconds to complete the same simulation.
  98\% of the runtime could be removed, simply by modernizing the
  source code and utilizing compiler optimizations.
  The main result of the optimization process is the observation that
  idiomatic, modern and easy-to-read Fortran code results not only in
  a far better performing program, but also in a more maintainable
  one.
  Utilizing the compiler properly was the main reason for the
  increased performance.
\end{abstract}

\begin{keywords}
Scientific programming, serial optimization, molecular dynamics,
Fortran
\end{keywords}

\section{Introduction} % {{{
\label{sec:intro}

This paper documents the serial performance optimization conducted for
a molecular dynamics program written in the Fortran programming
language.
The program reads data from $n$ molecules as its input and iterates
a predefined amount of steps, which correspond to the progress of time
in the simulation.
Each step the data of the molecules (e.g.\ force, position in a three
dimensional Euclidean space, velocity) is updated.
The program counts collisions between molecules and writes the
data for each molecule of the simulation out after a certain interval
of iterations.
This interval is called a superstep.
The original version of the program is not optimized for computational
efficiency.

This paper describes, documents and discusses the process of
optimizing the original version of the program serially.
The program was optimized for the Cirrus supercomputer, a tier-2
UK national supercomputer of the engineering and physical sciences
research council, which is hosted and maintained by the EPCC
\citep{cirrus}.
The compiler used was Intel's Fortran compiler \texttt{ifort},
version $18.0.5$ for the Linux operating system
\citep{ifort18}.

The conducted optimizations range from choosing the appropriate
compiler flags over rewriting performance critical sections of the
program to hardware specific optimizations, like leveraging
vectorization and cache optimizations.
Since raw performance benefits are not all that is important for
writing well performing and good programs, the maintainability and
portability of the program are also looked at and discussed.

The original version of the program takes 1270 seconds to complete the
simulation used for benchmarking.
The final version takes only 26 seconds to complete the same
benchmark.
98\% of the runtime was removed by the optimization efforts described
in this paper.
The main result of this paper is the observation that idiomatic,
modern and easy-to-read Fortran code results in the best performing
program, mostly by enabling the compiler to optimize effectively.

Not only is the performance of the program drastically improved,
maintainability is also highly increased.
Approximately 20\% of the source code got removed, making it easier
to follow what the program does.
Only in one case was code explicitly added to increase performance
(OpenMP's simd directive).
On all other occasions performance was increased by replacing badly
readable and inefficient code with modern Fortran code, killing two
birds with one stone.

This paper continues in Section~\ref{sec:md} with describing the
molecular dynamics program in detail.
Section~\ref{sec:setup} describes Cirrus and how the correctness of
the program is tested with a regression test suite.
Also the benchmark suite used for assessing the performance benefits
of an optimization is outlined, as is the simulation setup used for
benchmarking.
Section~\ref{sec:opt} lists, describes and discusses all
optimizations tested with their performance benefits.
Afterwards, the results are discussed in Section~\ref{sec:dis}.
At least a conclusion is drawn in Section~\ref{sec:con}.

% }}}

\section{Molecular Dynamics Program} % {{{
\label{sec:md}

This section describes what the program does.
The program is a simple molecular dynamics program.
It reads data from $n$ particles from a file.
In the following, subscripts $1 \leq i \leq n$ refer to a particle
of the simulation.
The data for a particle $i$ is its mass $m_i$, the viscosity of
a fluid $i$ is part of $vis_i$, the position of the particle's center
in a three dimensional space $\vec{p}_i$ and its velocity $\vec{v}_i$.

The program iterates a predefined amount of iterations.
Each iteration represents a time step in the particle simulation.
The time is updated by a constant value $\Delta_t$.
Every step the position and the velocity of each particle is
updated, based on the gravitational forces operating on each particle.
The gravitational forces are the forces between each particle and the
gravitational force coming from a large central mass located at the
origin of the Euclidean space.
The particles are part of a fluid, which means the viscosity of
the fluid must also be taken into account.
The last external force operating on a particle is wind $\vec{w}$,
which is a constant vector over the whole simulation.

The gravitational forces are computed based on Newton's law of
universal gravitation.
It states, that any two physical objects attract each other
with a force, which is proportional to the mass of both objects and
inversely proportional to the squared distance between both
\citep{feynman1963}.
The scalar gravitational force $F$ between two objects 1 and 2 can be
mathematically described as:
\begin{align}
  F_{1,2} = G\frac{m_1 m_2}{||\vec{p}_1 - \vec{p}_2||_2^2},
\end{align}
where $G$ is the gravitational constant, $m_i$ the mass of object $i$
and $||\vec{p}_1 - \vec{p}_2||_2$ the Euclidean or $L_2$ distance
between the centers of both objects.
The vector form of the gravitational force object 2 operates on object
1, which also accounts for the direction of the force, is given by:
\begin{align}
  \label{eq:F}
  \vec{F}_{1\leftarrow2} = -F_{1,2}\frac{\vec{p}_1 - \vec{p}_2}
                   {||\vec{p}_1 - \vec{p}_2||_2}
          = -G\frac{m_1 m_2 (\vec{p}_1 - \vec{p}_2)}
                   {||\vec{p}_1 - \vec{p}_2||_2^3}.
\end{align}
The gravitational force, which object 1 operates on object 2 is the
additive inverse of $\vec{F}_{1 \leftarrow 2}$:
$\vec{F}_{2 \leftarrow 1} = -\vec{F}_{1 \leftarrow 2}$.
If particles collide, the gravitational forces between the two object
are negated.
The program finds collisions, by checking if the distance of two
particles is smaller than a threshold $\tau_{1,2}$, which is the sum
of the radii of the two particles.
For the program, all particles have a radius of $\frac{1}{2}$, so the
threshold for a collision is $1$.
Now we can define a pairwise gravitational forces function for the
particles of the simulation as:
\begin{align}
  \label{eq:f}
  f_{1 \leftarrow 2} := \begin{cases}
    \vec{F}_{1 \leftarrow 2} &\text{if }
      ||\vec{p}_1 - \vec{p}_2||_2 \ge \tau_{1,2} \\
    -\vec{F}_{1 \leftarrow 2} &\text{otherwise}
  \end{cases}.
\end{align}
Other than the pairwise gravitational forces between the particles,
there is the gravitational force of the central mass
$\vec{F}_{1 \leftarrow central}$.
The central mass lies at the origin of the three dimensional space,
which means its position vector $\vec{p}_{central} = 0$.

The viscosity of the fluid is another force operating on a particle.
It is simply the negative of the viscosity of the fluid multiplied
by the velocity vector of particle $i: -vis_i \vec{v}_i$.
Shear velocity of the fluid is not taken into account.
Lastly, there is the wind force, which is simply the negated viscosity
of the fluid particle $i$ is part of multiplied by the wind vector:
$-vis_i \vec{w}$.
The overall force per iteration operating on particle $i$ can now
be described as:
\begin{align}
  \label{eq:F_all}
  \vec{F}_i = -vis_i(\vec{v}_i + \vec{w}) +
              \vec{F}_{i \leftarrow central} +
              \sum_{j\neq i}^n f_{i \leftarrow j}.
\end{align}
Based on $\vec{F}_i$ we can now update $\vec{p}_i$ and $\vec{v}_i$:
\begin{align}
  \label{eq:p}
  \vec{p}_i &= \vec{p}_i + \Delta_t \vec{v}_i \\
  \label{eq:v}
  \vec{v}_i &= \vec{v}_i + \Delta_t \frac{\vec{F}_i}{m_i}.
\end{align}
The iterations of the program are broken down into supersteps.
On completion of a superstep, the updated particles are exported to a
file with the same format as the input file.

% }}}

\section{Setup} % {{{
\label{sec:setup}

This section outlines the settings, under which the program was
optimized for performance. Information about the used hardware is
given.
The way correctness of the program was tested is described.
Lastly the settings and the criterion for benchmarking the
computational performance are presented.

Like stated in Section~\ref{sec:intro}, the program was optimized for
the Cirrus supercomputer \citep{cirrus}.
Since we are running the program serially, we are not concerned with
the amount of nodes or the interconnect, but will focus on a single
compute node.
A single compute node of Cirrus contains two 2.1 GHz, 18-core Intel
Xeon E5-2695 processors (code name: Broadwell).
The processor supports the AVX2 vector instruction set \citep{avx2}.
Each processor is connected to 128 Gigabyte of memory.
Both processors are within a NUMA region, so 256 Gigabyte of memory
are actually at ones disposal \citep{cirrus_hardware}.
The compute node offers three levels of cache:
\begin{enumerate}
  \item 32 Kilobyte instruction and 32 Kilobyte of data cache
        (per core)
  \item 256 Kilobyte (per core)
  \item 45 Megabyte (shared)
\end{enumerate}

Testing the correctness of the program is not as straight-forward as
it seems at first glance.
Like stated in Section~\ref{sec:md}, after each superstep, the
updated particles are written out to file.
Comparing the output of the optimized version of the program with the
original one would be a sufficient test for correctness, if it were
not for floating point rounding errors.
These accumulate and after a certain amount of time, the numbers
generated by the optimized version will be too different from the
original ones.

In order to avoid getting different results, just because of floating
point rounding errors, the program was augmented by a special test
setting.
This test setting differs from the normal program, because it reads
the data it has written out after a superstep back in.
That way, the next superstep will work with the floating point numbers
that are crippled by writing them to file.
The floating point numbers are written to file text based in
exponential form with 16 digits, eight digits on the right side of the
decimal point \citep[see e.g.][for formatting IO in Fortran]
{fortran_formats}.
Changing the output format to a more precise representation is not
possible, because this would mean the files generated as output would
not have the same format as the input file with the initial states of
the particles.
The test setting allows for effectively comparing the output files of
both versions, because floating point rounding errors now only
accumulate over a single superstep instead of the whole simulation.

Once the discrepancy between the output values of the optimized
version and the original one surpasses a predefined error level,
the optimization is deemed to result in an incorrect version of the
program.
The predefined error level was set to be $0.05$.
If any output file contains a \texttt{NaN} value, the program is
also deemed incorrect.
The regression test suite was implemented with a Python script.

The program runs five supersteps.
Each superstep encompasses 100 iterations.
Computation is done using double precision floats.
The input file which was used for optimizing contained 4096 particles.
The goal of the optimization process was to reduce the wall-clock
time of the program to a minimum, while bearing in mind portability
and more importantly maintainability.
The program measures the time it needs for each superstep and its
overall time, including the file output.
In order to build the benchmark suite around the program, the timings
are exported to another file when the program has finished the
simulation.
The program was benchmarked by running it ten times on a compute node
of Cirrus and taking the average from those ten runs as the
performance measurement.
Running it ten times is sufficient to get a stable average,
because running the program as a job on a compute node of Cirrus
means exclusive hardware access to that node.
Only IO performance can be influenced by other users, because Cirrus
uses Lustre for its file system which is shared
\citep{cirrus_hardware}.
As will be shown below, IO performance is actually negligible when
it comes to performance optimization when compared to the
computational effort of the simulation.

% }}}

\section{Optimizations} % {{{
\label{sec:opt}

This section documents the process of performance optimization of the
program.
Focus lies more on the process, not the results.
All the successively performed optimizations are described.
Code quality in form of readability, portability and maintainability
is taken into account during the whole process and the optimizations
are all looked at from this perspective.
The optimization process can basically broken down into five phases:
(\romannumeral 1) rewriting the source code to Fortran 90,
restructuring the source code without changing the critical section,
(\romannumeral 2) enabling basic compiler optimizations,
(\romannumeral 3) rewriting the program for better performance and
maintainability, (\romannumeral 4) taking hardware and environment
into account (mainly vectorization and memory access patterns) and
(\romannumeral 5) trying out more advanced compiler optimizations on
the rewritten version of the program again.

\subsection{First Phase} % {{{
\label{subsec:p1}

The first phase of optimization only concerns itself with increasing
the maintainability of the program.
The original version of the program is written in fixed format
Fortran \citep[see e.g.][for free vs.\ fixed format Fortran]
{fortran_free_fixed}.
Readability for screen based devices was deemed more of an issue than
formatting source code for punched cards, which are unfortunately not
supported by Cirrus.
So the first step was to reformat the source code to free format
Fortran to increase maintainability.

The original version of the program is spread across four files.
\texttt{control.f} contains the main program.
It performs initialization of the program.
This includes defining constants and reading the particles from the
initial file.
It contains the superstep loop and performs the output of the
intermediate states of the particles to file.
It also collects the timings for every superstep and the combined
time for all supersteps together.
The \texttt{MD.f} file contains the \texttt{evolve} subroutine.
This subroutine performs the main computations for the simulation.
It is called each superstep and iterates 100 times over the
simulation, updating the state of the particles.
The particle data is shared between the main program and the
\texttt{evolve} subroutine with a \texttt{COMMON} block
\citep[see e.g.][]{fortran_common}.
The \texttt{COMMON} block is defined in the \texttt{coord.inc} file,
which also contains the global constants $G$ and $m_{central}$.
Lastly, there is the \texttt{util.f} file containing utility
subroutines and functions, e.g.\ \texttt{visc\_force} or
\texttt{wind\_force}, which compute $-vis_i\vec{v}_i$ and
$-vis_i\vec{w}$ respectively (see Section~\ref{sec:md}).

Modern Fortran compilers like \texttt{ifort} version $18.0.5$ support
all Fortran 2008 features \citep{ifort18}.
Fortran introduced modules in Fortran 90, which make it much easier
to share data between subroutines and coupling can be much improved by
using them \citep{fortran_modules}.
Because using modules increases maintainability a lot, all routines
of the program are put into the main program in \texttt{control.f},
inside its \texttt{contains} block.
That way the particle data can be shared with the \texttt{evolve}
subroutine without a \texttt{COMMON} block.
These changes greatly increased maintainability, because of the
enhanced readability of the source code.
Also the build process is simplified, because only a single
Fortran 90 file needs to be compiled, rather than having to link
\texttt{control.f} and \texttt{MD.f} with \texttt{coord.inc}.

Both, the original version and the new Fortran 90 version were
modified to incorporate the test setting, where they read there
intermediate outputs back in in order to compute the next superstep.
In order for this setting not to interfere with the original setting
used for benchmarking, Intel's Fortran preprocessor was used
\citep{fpp}.
The additional reading back of the intermediate file is put into an
\texttt{\#ifdef} directive.
This way, the original version of the program used for benchmarking
is not damaged by additional checks at runtime.

Another aspect to consider in favor of the new version is the fact,
that the \texttt{COMMON} block is not aligned.
Figure~\ref{fig:common_unaligned} shows the compiler output, when
compiling the original version.
The \texttt{COMMON} block has alignment issues for the wind vector
$\vec{w}$, which could have an impact on the programs performance.
Removing the block removes the alignment issue.
So not only is the new version better maintainable, it also removes
the first performance issue with the original version.

Both versions were compiled using \texttt{ifort} version $18.0.5$ with
the following compiler flags which influence performance:
\texttt{-O0}, which disables any compiler optimization,
\texttt{-no-vec}, which inhibits vectorization and
\texttt{-check uninit,bounds}, which tells the compiler to add
extra instructions to the program which perform explicit checking
for uninitialized variables and out-of-bounds access of arrays.

Benchmarking the original version reveals that it takes on average
$1270$ seconds for all five supersteps to complete.
A single superstep takes on average $254$ seconds to complete.
If one subtracts the sum of the individual timings of all five
supersteps from the overall time, one gets the time spent doing the
file output.
The IO time lies at a quarter of a second for the original version,
which is $0.02\%$ of the overall runtime.
The new Fortran 90 version of the program takes only $1110$ seconds
on average to complete.
The average superstep time lies at $222$ seconds.
While the focus of the first phase of the optimization actually was
about enhancing maintainability and setting the right foundation for
the next phases, the performance was already increased by $12.5\%$.

\begin{figure}
\begin{verbatim}
ifort -g -O0  -check uninit,bounds -no-vec -fpp
-o ../bin/old_bench control.f MD.o util.o

./coord.inc(25): remark #6375: Because of COMMON, the alignment of
object is inconsistent with its type - potential performance
impact. [WIND]

      DOUBLE PRECISION wind(Ndim)
-----------------------^
\end{verbatim}
\caption{Output from \texttt{ifort} when compiling the original
  version of the program. The constant vector $\vec{w}$ is not
  aligned.}
\label{fig:common_unaligned}
\end{figure}

% }}}

\subsection{Second Phase} % {{{
\label{subsec:p2}

The second phase was about using the more common compiler flags to
enhance the performance of the Fortran 90 version.
The compiler flags used in phase one not only hinder compiler
optimizations with \texttt{-O0}, they even make the code perform worse
by adding the out-of-bounds access and uninitialized variables flags.
Therefore the first step to better performing code was to utilize the
compiler.
This phase was not about finding a definitive set of flags, but only
a first step to see how much the compiler can achieve using the
more common compiler flags for optimization.
It was more motivated by the still horrible runtime of 1110 seconds,
which hinders rapid development during the third phase.
Table~\ref{tab:p2} shows all the different flags tried and how they
impacted performance.

The first act was to remove the unnecessary checks for out-of-bounds
access of arrays and using uninitialized variables.
The Fortran 90 version of the code is riddled with loops, so removing
the checks for each should have quite the impact on performance.
As it turned out it did.
Removing the checks improved the average overall time by $33\%$.
That means $1/3$ of the time was spent checking for out-of-bounds
access and the use of uninitialized variables.

The next step was to gradually increase the level of compiler
optimization from \texttt{-O0} to \texttt{-O3}.
The first level of optimization \texttt{-O1} enables speed
optimizations that do not enlarge binary size.
Optimizations done include data-flow analysis, test replacement and
instruction scheduling.
\texttt{-O1} is designed for large codes with many branches that
are not loops \citep{o}.
While this description does not fit to the program at all, which
has only one significant branch (code structure is discussed below)
and spends most of its time in loops, \texttt{-O1} still increases
the performance by 44\%.
Average overall time is reduced from 734 seconds to 413 seconds
(see Table~\ref{tab:p2}).

\texttt{-O2} is the recommended level of compiler optimization.
It performs basic loop optimizations like interchanging, unrolling or
scalar replacements.
Furthermore inlining, intra-file ipo (interprocedural optimization),
dead code elimination and many more optimizations are enabled
\citep[see][]{o}.
Enabling \texttt{-O2} reduced the average overall time down to
85 seconds, which is 79\% better than the program compiled with
\texttt{-O1} (see Table~\ref{tab:p2}).

\texttt{-O3} enables more aggressive optimizations concerning loops
and memory access transformations, additionally to the optimizations
done using \texttt{-O2}.
Optimizations include loop fusion and collapsing if statements.
It is the recommended level of optimization for floating point
operation heavy programs that spend a lot of time in loops
\citep{o}.
This exactly describes the molecular dynamics program and
\texttt{-O3} actually increases the performance further.
The average overall time was further reduced from 85 seconds down
to 62 seconds.
This is an additional 27\% improvement over \texttt{-O2}
(see Table~\ref{tab:p2}).

At this point, vectorization was still disabled with the
\texttt{-no-vec} flag.
Enabling vectorization on the yet unoptimized Fortran 90 program
resulted in a regression of the average overall time of 33\%
(see Table~\ref{tab:p2}).
Like described above, the second phase is only about finding a
set of compiler optimizations that would enable a more rapid analysis
of the performance in the crucial third phase.
This is the reason why the drop in performance was not analyzed
further.
For the unoptimized Fortran 90 version of the program somehow
vectorization seems to cancel optimizations from \texttt{-O3}.
The guess at this point was that the program profits more from
optimizations from \texttt{-O3} (like loop unrolling) than
from being vectorized.

Lastly other common flags for compiler optimization were considered.
Two recommended options besides \texttt{-O3} are \texttt{-ipo} and
\texttt{-xHOST} \citep{user389}.
Neither would improve the performance of the program.
\texttt{-ipo} enables interprocedural optimizations between files
\citep{ipo}.
The program only consists of a single file, so \texttt{-ipo} would
not improve performance.
\texttt{-xHOST} forces the compiler to generate instructions from the
highest instruction set supproted by the host \citep{xhost}.
The host is a frontend node of Cirrus in this case.
The frontend nodes of Cirrus are the same as its compute nodes.
The highest instruction set supported is AVX2
(see Section~\ref{sec:setup}).
\texttt{-no-vec}, which is still enabled at this point, cancels out
\texttt{-xHOST}.

Lastly \texttt{-Ofast} was tested.
\texttt{-Ofast} is a shorthand compiler flag, which combines
\texttt{-O3} with a faster floating point model than the default one.
It sets \texttt{-O3}, \texttt{-no-prec-div} and
\texttt{-fp-model fast=2} \citep{ofast}.
\texttt{-no-prec-div} increases the speed of floating point divisions.
The cost of this flag is a reduction in precision \citep{no_prec_div}.
\texttt{-fp-model fast=2} works the same way.
It increases the performance on the cost of less precise results of
floating point operations \citep{fp_model}.
Enabling \texttt{-Ofast} does not result in less accurate test
results.
The program's correctness is still given, even though floating point
precision was lowered.
\texttt{-Ofast} does not increase the performance compared to
\texttt{-O3} (see Table~\ref{tab:p2}).

Phase two was terminated at this point.
The best compiler flags determined were \texttt{-O3} with
\texttt{-no-vec}.
The average overall time after phase two is 62 seconds.
This is an improvement of a staggering 94\% over the results after
phase one (1110 seconds), simply by enabling compiler optimization
and removing unnecessary checks.

\begin{table}
  \begin{tabu}{|l|X|l|X|X|}
    \hline
    Optimization &$\emptyset$ overall time
                 &$\emptyset$ superstep time  &$+/-\%$ &Status \\
    \hline
    Removed checks &734s &147s &33\% &Improvement \\
    \hline
    \texttt{-O1} &413s &83s &44\% &Improvement \\
    \hline
    \texttt{-O2} &85s &17s &79\% &Improvement \\
    \hline
    \texttt{-O3} &62s &12s &27\% &Improvement \\
    \hline
    Removed \texttt{-no-vec} &82s &16s &-33\% &Regression \\
    \hline
    \texttt{-Ofast} &62s &12s &0\% &Invariant \\
    \hline
  \end{tabu}
  \caption{Compiler flags tried during the second phase of
    optimization. The $+/-\%$ column displays the
    variation in average overall time from the best version of the
    program so far. For example, the best version for the removal of
    the extra checks was the Fortran 90 version from phase one.
    For \texttt{-Ofast} the best version of the code was the one
    compiled without the checks and with \texttt{-O3}.
  }
  \label{tab:p2}
\end{table}

% }}}

\subsection{Third Phase} % {{{
\label{subsec:p3}

\begin{algorithm} % {{{
  \caption{: original computation per time step}
  \label{alg:old}

  \begin{algorithmic}[1]
    \FOR{i=1,\dots,n}
      \STATE{$\vec{F}_i$ := $-vis_i \vec{v}_i$}
      \COMMENT{Compute viscosity force for particle $i$}
    \ENDFOR

    \FOR{i=1,\dots,n}
      \STATE{$\vec{F}_i$ := $\vec{F}_i - vis_i \vec{w}$}
      \COMMENT{Compute wind force for particle $i$}
    \ENDFOR

    \FOR{i=1,\dots,n}
      \STATE{$r_i$ := $||\vec{p}_i||_2$}
      \COMMENT{$r_i$ is used in loop below for the denominator in
        $\vec{F}_{i \leftarrow central}$}
    \ENDFOR

    \FOR{i=1,\dots,n}
      \STATE{$\vec{F}_i$ := $\vec{F}_i +
        \vec{F}_{i \leftarrow central}$}
      \COMMENT{see Equation~\ref{eq:F}}
    \ENDFOR
    \STATE{Compute pairwise forces with
      Algorithm~\ref{alg:pairwise_old}}

    \FOR{i=1,\dots,n}
      \STATE{$\vec{p}_i$ := $\vec{p}_i + \Delta_t \vec{v}_i$}
      \COMMENT{Update position vector of particle $i$
        (see Equation~\ref{eq:p})}
    \ENDFOR

    \FOR{i=1,\dots,n}
      \STATE{$\vec{v}_i$ := $\vec{v}_i + \Delta_t
        \vec{F}_i / m_i$}
      \COMMENT{Update velocity vector of particle $i$
        (see Equation~\ref{eq:v})}
    \ENDFOR

  \end{algorithmic}
\end{algorithm} % }}}

While the second phase already improved the performance by 94\%,
the quality of the program is still bad.
Both in consideration of performance and more importantly
maintainability.
Phase three of the optimization efforts therefore tackles this
problem by rewriting the critical section of the program, which spans
the computations for updating the particles at each time step.

The common workflow for optimizing for speed normally consists of
iterations, where the program is profiled, the bottleneck determined
and then optimized.
The workflow used here is less vigorous concerning profiling.
Profiling after each change in order to identify and fix bottlenecks
is not necessary, because it is well established where the bottleneck
of molecular dynamics programs is: computing the pairwise forces
between the particles \citep{chiu_et_al_2011}.

This assumption was validated for this program.
The update operation of the simulation was split into two parts:
(\romannumeral 1) computing pairwise forces between the particles and
(\romannumeral 2) computing the other forces and updating the position
and velocity of each particle.
Inlining was disabled and the program run with Intel's VTune 19
profiler \citep{vtune}.
The profiler revealed that over 99\% of the runtime is spend computing
the pairwise forces.

Furthermore, as described above, the guiding principle for rewriting
of the program was increased maintainability and elegance rather than
pure speed.
The codebase is not very big (approximately 250 lines of code) so not
a lot of time could be wasted on rewriting parts of the code that are
not part of the critical section.

The guiding tool for the environment specific optimizations was the
optimization report generated by the compiler with the
\texttt{-qopt-report=5} flag \citep{qopt_report}.
Otherwise the benchmark output and a Python script for comparing
different benchmarks with each other were used to determine the impact
a change to the source code had on the performance of the program.
The script basically replaced using an advanced profiler like VTune
after each change made to the code.

The first step rewriting the code was to remove the utility functions
and subroutines and inline them by hand (see Section~\ref{subsec:p1}).
This was done not in consideration of performance, but to better
enable the rewriting.
The utility functions were obscuring loops, making it harder to
properly see loop nests which are crucial for vectorization.
Looking at the optimization report revealed, that they were inlined
by the compiler, so no performance difference was measured after
the hand inlining.

Algorithm~\ref{alg:old} shows the update operation per time step,
as it was originally implemented.
As stated above, computing $\vec{F}_i$ without the pairwise
forces (Equation~\ref{eq:F_all} without
$\sum_{i \neq j}^n f_{i \leftarrow j}$) is not the critical section.
Nonetheless the code is still badly written and not optimal.
Algorithm~\ref{alg:old}, lines 1--12 are all needed, simply for
computing Equation~\ref{eq:F_all} without the pairwise forces.
While \texttt{-O3} enables loop fusion, only the first two loops
(the wind and viscosity forces) are fused together.
So there are three iterations needed over $n$: one for the wind and
viscosity forces, one for computing $r$ and one for computing
$\vec{F}_{i \leftarrow central}$.
The first step was to reduce the $3n$ iterations to $2n$ iterations by
fusing the viscosity, wind and central force computations all into
a single loop setting $\vec{F}_i := -vis_i(\vec{v}_i + \vec{w}) +
  \vec{F}_{i \leftarrow central}$.
Fusing the loops together by hand resulted in a regression concerning
the program's performance.
While the best version after phase two took on average 62 seconds
overall, the version with the fused loops resulted in 64 seconds
on average.
This is a drop in performance of approximately 3\%.
The reason for this drop are reduced efficiency of the pipelines of
the CPU caused by data hazards.
The data hazards are caused by output dependencies (write after write)
when computing $\vec{F}_i$ \citep[see e.g.][]{patterson_2014}.

\begin{figure} % {{{
  \begin{lstlisting}[language=Fortran]
    ! This is how the old version of the program computes r
    do k = 1, Nbody
      r(k) = 0.0
    end do
    do j = 1, Ndim
      do i = 1, Nbody
        r(i) = r(i) + pos(i,j) * pos(i,j)
      end do
    end do
    do k = 1, Nbody
      r(k) = sqrt(r(k))
    end do

    ! The new version uses Fortran's array syntax to convert this
    ! to a 1-liner
    r(:) = sqrt(sum(pos(:,:) ** 2, dim=2))

  \end{lstlisting}
  \caption{Computing $r$ for each particle in Fortran, as done in the
    old version of the program and the one using array syntax.}
  \label{fig:r}
\end{figure} % }}}

The next step was to beautify the computation of $r$
(Algorithm~\ref{alg:old}, lines 7--9).
Figure~\ref{fig:r} shows the original code and the one-liner it
was transformed to.
Using Fortran's array syntax makes the code much more readable by
being more concise and removing ten lines of code.
Unfortunately, making this change results in another performance
regression of about 1\%.
The reason for the performance drop is the fact that \texttt{pos(:,:)}
copies the content of the \texttt{pos} matrix to a new temporal one.
\texttt{pos} is the $n \times 3$ double precisision matrix storing the
particle vectors $\vec{p}_i$.
It occupies 96 Kb of memory.
Copying 96 Kb is a costly operation which is why the code is slower
than before.

In order to further concise the program, reduce the amount of loops
and the memory footprint, $r$ was removed from the program completely,
moving the computation of $||p_i||_2$ into the actual denominator of
$\vec{F}_{i \leftarrow central}$.
Problematic for this operation was the memory layout of the particle
vectors.
All particle vectors involved (force $\vec{F}_i$, position $\vec{p}_i$
and velocity $\vec{v}_i$) are actually represented as a $n \times 3$
matrix of all particles.
Fortran stores matrices column-wise.
This means all particle vectors are actually strided with a separation
of $n$ and not contiguous in memory.
This does not leverage locality needed for utilizing the cache for
fast memory access when doing particle-wise operations, rather than
dimensional-wise (outer loop over the three dimensions of the particle
vectors).
The current version of the program actually computes $\vec{F}_i$
dimensional-wise, utilizing locality.
$r_i$ can only be computed particle-wise.
So in order to remove $r$ from the program the loops for computing
$\vec{F}_i$ were exchanged to being particle-wise (outer loop over the
particles instead of over the dimensions of each particle).
While this change removes leveraging locality, removing $r$ actually
reduces the pressure on the cache.
Like stated in Section~\ref{sec:setup}, the whole program uses
double precision floats.
For $n = 4096$, this means $r$ alone takes up 32 Kb---the whole L1
cache.
The performance report reveals that the inner loop over the three
dimensions of each particle is actually unrolled by the compiler.
Removing $r$ from the program resulted in a performance improvement of
approximately 2\%, still worse than the version of the program after
phase two.

The last change made to the part of the simulation not dealing with
the computation of the pairwise forces was fusing the loop computing
$\vec{F}_i$ with the last two loops (Algorithm~\ref{alg:old}, lines
14--19), which were already fused.
$2n$ were reduced to $n$ iterations by fusing the two loops by hand.
Again this reduction in the constant factor of $n$ comes with the
cost of locality when doing particle-wise operations instead of
dimensional-wise ones.
Concerning the maintainability, the whole rewrite of the program to
this point saved approximately 30 lines of code and increased the
readability drastically.
Only a single loop is needed for computing $\vec{F}_i$ without the
pairwise forces and updating $\vec{p}_i$ and $\vec{v}_i$.
Fusing the two loops is invariant concerning the performance.

\begin{algorithm} % {{{
  \caption{: original pairwise forces computation}
  \label{alg:pairwise_old}

  \begin{algorithmic}[1]

    \FOR{i=1,\dots,n}
      \FOR{j=i+1,\dots,n}
        \STATE{$\vec{p}_{i,j}$ := $\vec{p}_i - \vec{p}_j$}
        \COMMENT{$\vec{p}_{i,j}$ is used in loop below and for the
          numerator in $f_{i \leftarrow j}$}
      \ENDFOR
    \ENDFOR

    \FOR{i=1,\dots,n}
      \FOR{j=i+1,\dots,n}
        \STATE{$\Delta p_{i,j}$ := $||\vec{p}_{i,j}||_2$}
        \COMMENT{$\Delta p_{i,j}$ is used in loop below for the
          denominator in $f_{i \leftarrow j}$}
      \ENDFOR
    \ENDFOR

    \FOR{i=1,\dots,n}
      \FOR{j=i+1,\dots,n}
        \STATE{$\vec{F}_i := \vec{F}_i + f_{i \leftarrow j}$}
        \COMMENT{see Equation~\ref{eq:f}}
        \STATE{$\vec{F}_j := \vec{F}_j + f_{j \leftarrow i}$}
        \COMMENT{see Equation~\ref{eq:f}}
      \ENDFOR
    \ENDFOR

  \end{algorithmic}
\end{algorithm} % }}}

Next the critical section of the program---computing the pairwise
forces $f_{i \leftarrow j}$---was optimized.
How the original program computes the pairwise forces is shown in
Algorithm~\ref{alg:pairwise_old}.
First two minor simplifications of the loop nest computing the
pairwise forces (Algorithm~\ref{alg:pairwise_old}, lines 11--16)
were conducted.
Two particles $i$ and $j$ collide, if $||\vec{p}_i - \vec{p}_j||_2 <
\tau_{i,j} := radius_i + radius_j$ (see Section~\ref{sec:md}).
$radius_i$ is never read from the file containing the data for each
particle.
Instead $\forall i: radius_i := 1/2$ is set in the routine reading
the particle data from file.
This way $\forall i,j: \tau_{i,j} = 1$, so we can remove $radius$ from
the program (saving another 32 Kb of memory) and replace it with a
single constant.
The second minor change concerns an unnecessary branch.
The program counts particle collisions.
Whenever $||\vec{p}_i - \vec{p}_j||_2 < 1$ a counter is increased.
Instead of increasing the counter in the if-statement of
Equation~\ref{eq:f}, a boolean variable is set to true and another
if-statement is needed for incrementing the collision counter.
The second if-statement was replaced with incrementing the collision
counter directly.
Again the changes increased readability but the program is 5\% slower
than the fastest version.

The next change was the first major change of phase three where
performance was significantly improved.
Instead of computing $\vec{p}_{i,j}$ and $\Delta p_{i,j}$ in their
own loops (Algorithm~\ref{alg:pairwise_old}, lines 1--10) for each
particle and having to save the result in memory for later use in
computing $f_{i \leftarrow j}$, they were reduced to two temporal
variables computed in the last loop (Algorithm~\ref{alg:pairwise_old},
lines 11--16).
There are $n(n-1)/2$ particle pairs.
This means removing the two loops results in a reduction of $n(n-1)$
iterations.
The original program actually saved $\vec{p}_{i,j}$ and
$\Delta p_{i,j}$ not in a matrix/array with $n(n-1)/2$ elements but
in oversized containers with $n^2$ elements.
This means the array for $\Delta p_{i,j}$ alone takes up 128 Mb
of memory.
The matrix which saves $\vec{p}_{i,j}$ takes up 384 Mb of memory.
The fastest version took on average 62 seconds to complete the whole
simulation.
The new version only takes on average 42 seconds to completion, a
performance improvement of 32\%.

The next two changes tried to reduce the impact of the branch in
Equation~\ref{eq:f}.
Computing $f_{i \leftarrow j}$ was changed to a particle-wise
operation (see above).
This way the branch was moved up one loop in the nest and the inner
loop over the dimensions was again unrolled.
This resulted in a 2\% performance improvement.
At this point $f_{i \leftarrow j}$ and $f_{j \leftarrow i}$ were
still both computed, even though
$f_{j \leftarrow i} = -f_{i \leftarrow j}$.
Readability is increased by using
$\vec{F}_j := \vec{F}_j - f_{i \leftarrow j}$ instead of
$\vec{F}_j := \vec{F}_j + f_{j \leftarrow i}$.
This change has no effect on the performance.

% }}}

\subsection{Fourth Phase} % {{{

\begin{algorithm} % {{{
  \caption{: computation per time step after phase three}
  \label{alg:new}

  \begin{algorithmic}[1]

    \FOR{i=1,\dots,n}
      \FOR{j=i+1,\dots,n}
        \STATE{$\vec{F}_i := \vec{F}_i + f_{i \leftarrow j}$}
        \COMMENT{see Equation~\ref{eq:f}}
        \STATE{$\vec{F}_j := \vec{F}_j - f_{i \leftarrow j}$}
        \COMMENT{see Equation~\ref{eq:f}}
      \ENDFOR
    \ENDFOR

    \FOR{i=1,\dots,n}
      \STATE{$\vec{F}_i := \vec{F}_i -vis_i(\vec{v}_i + \vec{w})
        + \vec{F}_{i \leftarrow central}$}
      \STATE{$\vec{p}_i := \vec{p}_i + \Delta_t \vec{v}_i$}
      \STATE{$\vec{v}_i := \vec{v}_i + \Delta_t \vec{F}_i/m_i$}
    \ENDFOR

  \end{algorithmic}
\end{algorithm} % }}}

The rewrite in phase three was necessary not only for increasing
maintainability, but also to make the goal of this phase easier to
achieve.
This phase concerns itself with hardware and environment specific
optimizations, mainly enabling vectorization.
As will be shown below, idiomatic and modern Fortran code makes it
easy for the compiler to auto-vectorize and only minor efforts were
needed to enforce vectorization where the compiler could not
auto-vectorize.

Algorithm~\ref{alg:new} shows how the particles are updated for each
time step after the optimizations in phase three.
Up to this point vectorization was still disabled with the
\texttt{-no-vec} flag.
The first step of phase four was to enable vectorization.
The compiler vectorizes the code when the \texttt{-no-vec} flag is
not passed.
But in order to guide the compiler to fully utilize vectorization,
\texttt{-no-vec} was replaced with the \texttt{-xCORE-AVX2} and the
\texttt{-vec-threshold0} flags.
\texttt{-xCORE-AVX2} tells the compiler to generate AVX2 instructions
when possible and optimize the code for this instruction set
\citep{xcore_avx2}.
\texttt{-vec-threshold0} tells the compiler to vectorize whenever
possible \citep{vec_threshold0}.

The second loop (Algorithm~\ref{alg:new}, lines 7--11)
is automatically vectorized by the compiler.
The inner loops over the three dimensions of the vectors are unrolled.
The compiler tries to vectorize the inner most loop in a loop nest.
It is unable to vectorize the inner loop for computing the pairwise
forces (ranging over $j=i+1,\dots,n$), because of data dependencies
(flow and anti dependency for updating $\vec{F}_i$).
Like stated in Section~\ref{subsec:p3}, the pairwise forces loop takes
over 99\% of all the runtime of the program.
Therefore it is not surprising that the vectorized second loop does
not speed-up the program.
Without making the pairwise loop vectorize, enabling vectorization is
not increasing the performance.

The inner loop of computing the pairwise forces was vectorized using
OpenMP 4.5's simd directive \citep{omp}.
The outer loop could not be properly vectorized, because of the
irregularity of the inner loop, it depending on $i$.
This hindered OpenMP to successfully collapse the inner and outer
loops.
In order to cope with the flow and anti dependencies $\vec{F}_i$ was
removed from the inner loop and replaced with a temporal reduction
variable $\vec{F}^\prime_i$.
After the inner loop completed the temporal variable was integrated
into $\vec{F}_i$ by setting
$\vec{F}_i := \vec{F}_i + \vec{F}^\prime_i$.
Using the reduction clause of the simd directive over
$\vec{F}^\prime_i$, the inner loop was vectorized successfully.
The average overall runtime of the program was reduced from 42 seconds
to 37 seconds, an increase in performance of 12\%.

Looking at the performance report revealed that the compiler thinks
accessing the vectors $\vec{p}_j$ and $\vec{F}_j$ is unaligned.
Telling the compiler that these are actually aligned accesses (with a
stride of $n$) with the \texttt{!dir\$ vector aligned} compiler hint
did not increase the performance.
All other array and matrix accesses are automatically aligned.

One major part of the optimization strategy was reducing loops.
The last step to reduce the amount of loops in the program to the
minimal amount was to fuse the outer loop of computing the
pairwise forces (Algorithm~\ref{alg:new}, line 1) with the update loop
(Algorithm~\ref{alg:new}, line 7).
This destroys the vectorization of the update loop while reducing the
amount of iterations performed by the program.
But instead of unrolling the dimensional-wise vector operations
(Algorithm~\ref{alg:new}, lines 8--10), they were vectorized by the
compiler.
Fusing the two loops did not increase the performance.
Algorithm~\ref{alg:final} shows the structure of the final version of
the update operation executed for each time step of the simulation.

At this point all dimensional-wise operations were replaced with
particle-wise.
The dimensions of the matrices containing the particle vectors are
still $n \times 3$, so the dimensions of the particle vectors are
stored contiguously in memory.
This means particle-wise operations are strided with a stride of $n$,
reducing the utilization of the cache (see Section~\ref{subsec:p3}).

The last part of the rewriting process tried to increase the cache
utilization by transposing the matrices containing the particle
vectors to remove the stride from particle-wise operations.
Transposing the matrices was done after vectorization was enabled.
First enabling vectorization was done, because this way possible
alignment issues with the transposed version are exposed.
Vectorization works best with aligned base pointers.
For Cirrus's Broadwell processor (see Section~\ref{sec:setup}) the
data is best aligned on 64 byte boundaries \citep{krishnaiyer2015}.
A particle vector is only 24 bytes long.
Accessing every particle except the first is not aligned based on the
64 byte boundary.
As expected, making particles contiguous in memory reduced the
performance by 5\%, from 38 seconds to 40 seconds, because accessing
the particles is not aligned anymore.
The processor can handle strided loads and stores better than it can
handle unaligned accesses.

\begin{algorithm} % {{{
  \caption{: final computation per time step}
  \label{alg:final}

  \begin{algorithmic}[1]

    \FOR{i=1,\dots,n}

      \FOR{j=i+1,\dots,n}
        \STATE{$\vec{F}^\prime_i := \vec{F}^\prime_i
          + f_{i \leftarrow j}$}
        \COMMENT{see Equation~\ref{eq:f}}
        \STATE{$\vec{F}_j := \vec{F}_j - f_{i \leftarrow j}$}
        \COMMENT{see Equation~\ref{eq:f}}
      \ENDFOR

      \STATE{$\vec{F}_i := \vec{F}_i + \vec{F}^\prime_i
        -vis_i(\vec{v}_i + \vec{w}) + \vec{F}_{i \leftarrow central}$}
      \STATE{$\vec{p}_i := \vec{p}_i + \Delta_t \vec{v}_i$}
      \STATE{$\vec{v}_i := \vec{v}_i + \Delta_t \vec{F}_i/m_i$}
    \ENDFOR

  \end{algorithmic}
\end{algorithm} % }}}

% }}}

\subsection{Fifth Phase} % {{{
\label{subsec:p5}

The fifth and last phase was again about using the compiler to
increase the performance.
To this point the program was compiled with the following flags:
\texttt{-fpp}, \texttt{-qopenmp}, \texttt{-O3}, \texttt{xCORE-AVX2}
and \texttt{-vec-threshold0}.
Only the last three flags are relevant for performance, while the
first two just enable features used in the program (Fortran
preprocessor and OpenMP).
Some more aggressive compiler optimizations were tried to further
increase the performance.
Table~\ref{tab:p5} lists the compiler flags tested and their impact on
the program's performance.

The best version of the program at this point took on average 37
seconds to complete the simulation.
\texttt{-Ofast} was tried in phase two.
It did not increase performance on the unoptimized version of the
program.
On the optimized version of the program on the other hand
\texttt{-Ofast} results in improved performance.
Ten seconds are cut off of the simulation's overall runtime. That
is an improvement of 27\%.

\texttt{-Ofast} makes additional performance optimization to the
floating point model.
Because they worked so well two additional compiler optimizations for
floating point speed were tried: \texttt{-fp-speculation=fast} and
\texttt{-fma}.
\texttt{-fp-speculation=fast} enables floating point speculation which
can relax the way floating point operations are carried out
\citep{fp_speculation}.
\texttt{-fma} tells the compiler to use vectorized fused multiply-add
instructions where applicable \citep{fma}.
Neither enhances the performance of the program.

Next the maximum level of software prefetching was enabled in order to
increase the cache hit-rate.
This is done with the \texttt{-qopt-prefetch=5} flag
\citep{qopt_prefetch}.
This resulted in a 5\% performance improvement, taking away another
second of average overall time.

Lastly the compiler was told to optimize the memory structure of the
program.
This was done with the \texttt{-qopt-mem-layout-trans=3} and the
\texttt{-pad} flags \citep{qopt_mem_layout_trans, pad}.
Neither benefited the program.

Phase five was the last phase of the optimization process.
Both maintainability and performance of the program were significantly
improved.
Approximately 20\% of the source code was removed.
A test setting which enables regression tests was added.
The program was successfully vectorized and the memory usage was cut
down drastically.
The unoptimized version of the program took on average 1270 seconds
to finish the simulation described in Section~\ref{sec:setup}.
The final version took 26 seconds on average.
98\% of the time the original program took for the simulation was
successfully removed.

\begin{table}
  \begin{tabu}{|X|l|l|l|l|}
    \hline
    Optimization &$\emptyset$ overall time
                 &$\emptyset$ superstep time  &$+/-\%$ &Status \\
    \hline
    \texttt{-Ofast} &27s &5s &27\% &Improvement \\
    \hline
    \texttt{-fp-speculation=fast} &27s &5s &0\% &Invariant \\
    \hline
    \texttt{-fma} &27s &5s &0\% &Invariant \\
    \hline
    \texttt{-qopt-prefetch=5} &26s &5s &5\% &Improvement \\
    \hline
    \texttt{-qopt-mem-layout-trans=3} &26s &5s &0\% &Invariant \\
    \hline
    \texttt{-pad} &26s &5s &0\% &Invariant \\
    \hline
  \end{tabu}
  \caption{Compiler flags tried during the fifth phase of
    optimization (compare Table~\ref{tab:p2}).}
  \label{tab:p5}
\end{table}

% }}}

% }}}

\section{Discussion} % {{{
\label{sec:dis}

This section will discuss the most remarkable observations made during
the optimization process.
Most remarkably is the fact that well readable, idiomatic and modern
Fortran code that uses high level features like array slicing results
in a very well performing program which is also easy to maintain.
The biggest performance gains were made by properly utilizing the
compiler to make optimizations.
The biggest performance benefit observed of well written code
was the fact that it enables the compiler to properly optimize.

Table~\ref{tab:discussion} shows the average overall performance of
the best version of the program after each optimization phase.
Clearly the biggest performance gains were made in phase two.
Phase two enabled compiler optimization in the first place.
In phases three and four the program was rewritten to be more
maintainable and faster.
A lot of memory was saved, as was runtime.
Compared to the unoptimized version of the program which was used in
phase two, another performance gain of 40\% was achieved by rewriting
the program and enabling vectorization.
But more remarkably are the results of phase five.
The fifth phase was again about utilizing the compiler to increase
the performance further.
\texttt{-Ofast} was tried in phase two.
It did not increase the performance of the unoptimized version of the
program (see Section~\ref{subsec:p2}).
\texttt{-Ofast} on the optimized version on the other hand resulted in
a performance gain of 27\% (see Section~\ref{subsec:p5}).
Combining the raw speed gains of the optimized version with the
speed gains it enabled by allowing the compiler to properly optimize,
the optimized version gains 58\% on the unoptimized version of the
program.
This is a remarkable number and validates the rewriting of the program
as successful from a performance perspective.

Like stated above, not only is raw performance considered during the
optimization process, but also maintainability.
Maintainability is not as easy to measure as is performance.
The best measurement that can be presented is the fact, that the
optimized version is 20\% smaller concerning the lines of code than
is the original program.
The build process is also much simplified and only a single source
file is needed to the previous four.
Again does the optimized version leverage (relatively) modern Fortran
features like modules over old-fashioned ones like \texttt{COMMON}
blocks to increase its maintainability (see Section~\ref{subsec:p1}).

\begin{table}
  \begin{tabu}{|X|c|c|c|c|c|c|}
    \hline
      &Original &Phase one &Phase two &Phase three &Phase four
      &Phase five \\
    \hline
    $\emptyset$ overall time &1270s &1110s &62s &42s &37s &26s \\
    \hline
    $+/-\%$ &- &13\% &94\% &32\% &12\% &30\% \\
    \hline
  \end{tabu}
  \caption{Benchmark results after each optimization phase.}
  \label{tab:discussion}
\end{table}

% }}}

\section{Conclusion} % {{{
\label{sec:con}

The optimization process described in Section~\ref{sec:opt} is
considered to be successful from both perspectives: performance and
maintainability.
The original version of the program took on average 1270 seconds to
complete the simulation described in Section~\ref{sec:setup}.
The final version of the program took only 26 seconds to complete the
same simulation.
That is a performance gain of 98\%.

These gains were mainly due to the proper utilization of the compiler.
The most remarkable observation made during the optimization process
was the fact that idiomatic and modern Fortran code makes it
easy for the compiler to optimize, while also increasing the
maintainability of the program.

This paper only considered serial performance.
The obvious next step to further increase the program's performance
and enable much bigger simulations would be to parallelize it to
multiple threads (using e.g.\ OpenMP) or even distribute it to
multiple nodes (e.g.\ with MPI).

% }}}

\bibliography{library.bib}

\end{document}
