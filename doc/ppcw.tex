\documentclass[twoside,11pt]{article}
\PassOptionsToPackage{hyphens}{url}
\usepackage{jmlr2e}
\usepackage{amsmath}
\usepackage[toc,page]{appendix}
\usepackage[table]{xcolor}
\usepackage[marginparsep=30pt]{geometry}
\usepackage{stmaryrd}
\usepackage{algorithm}
\usepackage{algorithmic}
\usepackage{tikz}
\usepackage{pgfplots}
\usepackage{tabu}
\usepackage{longtable}
\usepackage{tabularx}
\usepackage{listings}
\usepackage{fancyref}
\usepackage{relsize}
\usepackage{float}
\usepackage{subcaption}
\usepackage{diagbox}

\usetikzlibrary{%
    arrows,
    arrows.meta,
    decorations,
    backgrounds,
    positioning,
    fit,
    petri,
    shadows,
    datavisualization.formats.functions,
    calc,
    shapes,
    shapes.multipart,
    matrix,
    plotmarks
}

\usepgfplotslibrary{fillbetween, statistics}

\pgfplotsset{
  compat=1.3,
  every non boxed x axis/.style={
  enlarge x limits=false,
  x axis line style={}%-stealth},
  },
  every boxed x axis/.style={},
  every non boxed y axis/.style={
  enlarge y limits=false,
  y axis line style={}%-stealth},
  },
  every boxed y axis/.style={},
}

\def\perc{\texttt{perco\-late}}
\def\v{\texttt{v0.1.0}}

\def\titl{Performance Programming Coursework:
  Serial Optimization of a Molecual Dynamics Program}

\title{\titl}

\author{}

\ShortHeadings{B160509}{B160509}
\firstpageno{1}


\begin{document}

\maketitle

\begin{abstract}
\end{abstract}

\begin{keywords}
Scientific programming, serial optimization, molecular dynamics,
Fortran
\end{keywords}

\section{Introduction} % {{{

This paper documents the serial performance optimization conducted for
a molecular dynamics program written in the Fortran programming
language.
The program reads data from $n$ molecules as its input and iterates
a predefined amount of steps.
Each step the data of the molecules (e.g.\ force, position in a three
dimensional space, velocity) is updated.
The program counts collisions between molecules and writes the
data for each molecule of the simulation out at certain, also
predefined, intervals.
The original version of the program is not optimized for computational
efficiency.

This paper describes, documents and discusses the process of
optimizing the original version of the program serially.
The program was optimized for the Cirrus supercomputer, a tier-2
UK national supercomputer of the engineering and physical sciences
research council, which is hosted and maintained by the EPCC
\citep{cirrus}.
The compilers used were Intel's Fortran compiler \texttt{ifort},
version $18.0.5$ and $19.0.0$ for the linux operating system
\citep{ifort18, ifort19}.
The conducted optimizations range from choosing the appropriate
compiler flags over rewriting performance critical sections of the
program to hardware specific optimizations, like leveraging
vectorization and cache optimizations.
Since raw performance benefits are not all that is important for
writing well performing and good programs, the maintainability and
portability of the program are also looked at and discussed.

% TODO: hint on the results

This paper continues in Section~\ref{sec:md} with describing the
molecular dynamics program in detail.
Section~\ref{sec:setup} describes Cirrus and how the correctness of
the program is tested with a regression test suite.
Also the benchmark suite used for assessing the performance benefits
of an optimization is outlined.
Section~\ref{sec:opt} lists, describes and discusses all
optimizations tested with their performance benefits.
Afterwards, the results are discussed in Section~\ref{sec:dis}.
At least a conclusion is given in Section~\ref{sec:con}.
% }}}

\section{Molecular Dynamics Program} % {{{
\label{sec:md}

This section describes what the molecular dynamics program, which is
optimized, does.
The program is a simple molecular dynamics program.
It reads data from $n$ particles from a file.
In the following, subscripts $1 \leq i \leq n$ refer to a particle
of the simulation.
The data for a particle $i$ is its mass $m_i$, the viscosity of
a fluid $i$ is part of $vis_i$, the position of the particle's center
in a three dimensional space $\vec{p}_i$, which is relative to a large
central mass and its velocity $\vec{v}_i$.
The program iterates a predefined amount of iterations.
Each iteration represents a time step in the particle simulation.
The time is updated by a constant value $\Delta_t$.
Every step the position and the velocity of each particle is
updated, based on the gravitational forces operating on each particle.
The gravitational forces are the forces between each particle and the
gravitational force coming from the central mass.
The particles are part of a fluid, which means the viscosity of
the fluid must also be taken into account.
The last external force operating on a particle is wind $\vec{w}$,
which is a constant vector over the whole simulation.

The gravitational forces are computed based on Newton's law of
universal gravitation.
It states, that any two physical objects attract each other
with a force, which is proportional to the mass of both objects and
inversely proportional to the squared distance between both
\citep{feynman1963}.
The scalar gravitational force $F$ between two objects 1 and 2 can be
mathematically described as:
\begin{align}
  F_{1,2} = G\frac{m_1 m_2}{||\vec{p}_1 - \vec{p}_2||_2^2},
\end{align}
where $G$ is the gravitational constant, $m_i$ the mass of object $i$
and $||\vec{p}_1 - \vec{p}_2||_2$ the Euclidean or $L_2$ distance
between the centers of both objects.
The vector form of the gravitational force object 2 operates on object
1, which also accounts for the direction of the force, is given by:
\begin{align}
  \vec{F}_{1\leftarrow2} = -F_{1,2}\frac{\vec{p}_1 - \vec{p}_2}
                   {||\vec{p}_1 - \vec{p}_2||_2}
          = -G\frac{m_1 m_2 (\vec{p}_1 - \vec{p}_2)}
                   {||\vec{p}_1 - \vec{p}_2||_2^3}.
\end{align}
The gravitational force, which object 1 operates on object 2 is the
additive inverse of $\vec{F}_{1 \leftarrow 2}$:
$\vec{F}_{2 \leftarrow 1} = -\vec{F}_{1 \leftarrow 2}$.
If particles collide, the gravitational forces between the two object
are negated.
The program finds collisions, by checking if the distance of two
particles is smaller than a threshold $\tau_{1,2}$, which is the sum
of the radii of the two particles.
For the program, all particles have a radius of $\frac{1}{2}$, so the
threshold for a collision is $1$.
Now we can define a pairwise gravitational forces function for the
particles of the simulation as:
\begin{align}
  f_{1 \leftarrow 2} := \begin{cases}
    \vec{F}_{1 \leftarrow 2} &\text{if }
      ||\vec{p}_1 - \vec{p}_2||_2 \ge \tau_{1,2} \\
    -\vec{F}_{1 \leftarrow 2} &\text{otherwise}
  \end{cases}.
\end{align}
Other than the pairwise gravitational forces between the particles,
there is the gravitational force of the central mass
$\vec{F}_{1 \leftarrow central}$.
The central mass lies at the origin of the three dimensional space,
which means its position vector $\vec{p}_{central} = 0$.

The viscosity of the fluid is another force operating on a particle.
It is simply the negative of the viscosity of the fluid multiplied
by the velocity vector of particle $i: -vis_i \vec{v}_i$.
Shear velocity of the fluid is not taken into account.
Lastly, there is the wind force, which is simply the negated viscosity
of the fluid particle $i$ is part of multiplied by the wind vector:
$-vis_i \vec{w}$.

The overall force per iteration operating on particle $i$ can now
be described as:
\begin{align}
  \vec{F}_i = -vis_i(\vec{v}_i + \vec{w}) +
              \vec{F}_{i \leftarrow central} +
              \sum_{j\neq i}^n f_{i \leftarrow j}.
\end{align}

Based on $\vec{F}_i$ we can now update $\vec{p}_i$ and $\vec{v}_i$:
\begin{align}
  \vec{p}_i &= \vec{p}_i + \Delta_t \vec{v}_i \\
  \vec{v}_i &= \vec{v}_i + \Delta_t \frac{\vec{F}_i}{m_i}.
\end{align}
The iterations of the program are broken down into supersteps.
On completion of a superstep, the updated particles are exported to a
file with the same format as the input file.

% }}}

\section{Setup} % {{{
\label{sec:setup}

% Cirrus

% test suite, benchmark suite

% }}}

\section{Optimizations} % {{{
\label{sec:opt}

% }}}

\section{Discussion} % {{{
\label{sec:dis}

% }}}

\section{Conclusion} % {{{
\label{sec:con}

% }}}

\bibliography{library.bib}

\end{document}
